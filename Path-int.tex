\documentclass[10pt]{jsarticle}
\usepackage[dvipdfmx]{graphicx}
\usepackage{braket}
\usepackage{bm,latexsym,amsmath,amssymb,amsfonts,mathrsfs}
\usepackage{tensor}
\usepackage{comment}
%\usepackage{physics}
\usepackage[dvipdfmx,linktocpage=true]{hyperref}
\usepackage{pxjahyper}
\usepackage{color}
\usepackage{here}
\usepackage{tikz}
\usetikzlibrary{arrows.meta}
\usetikzlibrary{bending}
\input{colordvi.tex}

\setcounter{tocdepth}{3}%目次にsubsubsectionまで入れる


\newcommand{\kakko}[1]{\left(#1 \right)} %丸括弧()
\newcommand{\kkakko}[1]{\left[ #1 \right]} %角括弧[]
\newcommand{\nkakko}[1]{\left\{ #1 \right\}} %波括弧{}
\newcommand{\p}[1]{\left| #1 \right|} %縦棒||
\newcommand{\henbi}[2]{{\frac{\partial#1}{\partial#2}}} %偏微分
\newcommand{\bi}[2]{{\frac{d #1}{d #2}}} %微分

\newcommand{\del}{{\partial}} %偏微分の∂
\newcommand{\skakko}[1]{$\ll$#1$\gg$}
\newcommand{\vc}[1]{\overrightarrow{#1}}% 矢印ベクトル
\newcommand{\vct}[1]{\bm{#1}}%太字ベクトル
\newcommand{\sint}[1]{\int\mathcal{D}#1\,}%経路積分
\newcommand{\pms}[1]{\mathcal{D}#1\,}%経路積分の測度
\newcommand{\Tr}{\mathrm{Tr\,}}
\newcommand{\tr}{\mathrm{tr\,}}

%\newcommand{\Res}{\mathrm{Res}}
\renewcommand{\Im}{\mathrm{Im\,}}
\renewcommand{\Re}{\mathrm{Re\,}}


\DeclareMathOperator*{\Res}{Res}
\DeclareMathOperator*{\ext}{ext}

\renewcommand{\theenumi}{\Alph{enumi}}
\renewcommand{\labelenumi}{(\theenumi)}

\numberwithin{equation}{section}%数式番号を(n.m)の形で書く。

\topmargin=0.0in
\headsep=0.0in
\headheight=0.0in
\oddsidemargin=-0.22in
\evensidemargin=-0.22in
\textwidth=6.5in
\textheight=9.0in
\title{経路積分の方法}
\author{あけみ@homuotaku}
\date{\today}
\pagestyle{empty}
\begin{document}
\maketitle
経路積分についてまとめる。
参考資料や私の気分によって、このノートでもノーテーションのブレがあるかもしれない。
\tableofcontents
\section{Faynnmanの考え}
2重スリット実験の帰結では、電子銃から放たれ、2つのスリットを通った粒子のスクリーン上での状態は、確率密度の重ね合わせで実現する。
そのスリットを3重にしたら、あるいはスリットをもう1枚増やしたらどうなるかというと、スクリーンに到達した電子の状態は、これもまたそれらの重ね合わせで表される。
ではスリットの穴と枚数をどんどん増やしていき、ついに全面が穴となり、また電子銃とスクリーンの間に隙間なくそのスリットが敷き詰められた場合を考える。
結局のところ、電子銃とスクリーンの間に何も無いのと同じなのだが、前述の議論を参考にすれば「あらゆる可能な経路を通る重ね合わせ」でスクリーン上の状態が実現するといえる。
言い換えれば、始状態と終状態の間の遷移確率振幅は、その2つを結ぶすべての経路の重ね合わせで与えられる、ということになる。
\section{量子力学において}
はじめにハミルトニアン$H$は時間並進の生成子であり、$U(t)=\exp{(-\frac{i}{\hbar}tH)}$は$t$だけ時間をシフトするユニタリ演算子であることを確認する。

時間並進の演算子$U(t,t_{0})$として、状態$\ket{\psi(t_{0})}$から$\ket{\psi(t)}$に時間発展させることは以下のように書ける。
\begin{equation}
\label{hatten}  \ket{\psi(t)}=U(t,t_0)\ket{\psi(t_{0})}
\end{equation}
この演算子は
\begin{align}
 \label{32} i\hbar \henbi{}{t}U(t,t_{0})&=H(t)U(t,t_{0}),\quad U(t_{0},t_{0})=I\\
  \label{33} U(t_{2},t_{1})U(t_{1},t_{0})&=U(t_{2},t_{0}),\quad(t_{2}>t_{1}>t_{0})
\end{align}
を満たす。これを示すために、Schrödinger方程式
\begin{equation}
  i\hbar \henbi{}{t}\ket{\psi(t)}=H(t)\ket{\psi(t)}
\end{equation}
を考える。(\ref{hatten})を代入すれば
\begin{equation}
  i\hbar \henbi{}{t}U(t,t_0)\ket{\psi(t_{0})}=H(t)U(t,t_0)\ket{\psi(t_{0})}
\end{equation}
ここで$\ket{\psi(t_0)}$は任意なので、例えば位置表示で$\ket{\psi(t_{0})}=\ket{q}$をし、さらに両辺$\bra{q}$をかけて積分することで完全性によって(\ref{32})が成立する。次にハミルトニアンが時間に依存しないとき
\begin{equation}
  U(t,t_{0})=\exp\kakko{-\frac{i}{\hbar} (t-t_0)H}
\end{equation}
と書けることを示す。時間を$N$等分することを考え、(\ref{32})の時間微分について無限小の$\Delta t=(t-t_{0})/N$を用いて書き直す。
\begin{equation}
  i\hbar \frac{U(t+\Delta t,t)-U(t,t)}{\Delta t}=H(t)U(t,t)
\end{equation}
$U(t,t)=I$なので
\begin{equation}
   U(t+\Delta t,t)=I-\frac{i}{\hbar}\Delta t H(t)
\end{equation}
さらに$t_{j}=t_{j-1}+\Delta t(j=1,2\cdots,N)$と書けば
\begin{equation}
  U(t_{j},t_{j-1})=I-\frac{i}{\hbar}\Delta t H(t_{j})
\end{equation}
これと(\ref{33})を繰り返し用いれば、
\begin{equation}
  U(t,t_{0})=\lim_{N\to \infty} U(t,t_{N-1})\cdots U(t_{1},t_{0})=\lim_{N\to \infty}\kakko{ I-\frac{i}{\hbar}\Delta t H(t) }\cdots \kakko{I-\frac{i}{\hbar}\Delta t H(t_{1})}
\end{equation}
と書ける。なお、$t_{N}=t$である。いまハミルトニアンは時間依存しない状況を考えているので$H(t)=H$と書いて、$\Delta t=(t-t_{0})/N$と表示すれば
\begin{equation}
  U(t,t_{0})=\lim_{N\to \infty}\kakko{I-\frac{i}{\hbar}\frac{t-t_0}{N}H }^{N}=\exp\kakko{-\frac{i}{\hbar}(t-t_{0})H}
\end{equation}
と書ける。
\subsection{Feynman核}
ここでは{\bf Feynman核}について議論する。Feynman核は
\begin{equation}
 \label{f-kn} K(q_{F},t_{F};q_{I},t_{I})=\braket{q_{F},t_{F}|q_{I},t_{I}}=\braket{q_{F}|e^{-\frac{i}{\hbar}(t_{F}-t_{I})H}|q_{I}}
\end{equation}
で与えられ、経路積分において基本となるものである。
これは時刻$t_{I}$、位置$q_{I}$の粒子が時刻$t_{F}$で位置$q_{F}$に見出される確率振幅に対応する。
Feynman核を用いて波動関数表示について
\begin{equation}
  \psi(q_{F},t_{F})=\int dq_{I}K(q_{F},t_{F};q_{I},t_{I})\psi(q_{I},t_{I})
\end{equation}
と書ける。これは状態を位置表示で
\begin{equation}
  \ket{\psi}=\int dq_{I}\ket{q_{I}}\braket{q_{I}|\psi}
\end{equation}
と書けることから両辺$\bra{q_{F}}$をかけて
\begin{equation}
  \braket{q_{F}|\psi}=\int dq_{I}\braket{q_{F}|q_{I}}\braket{q_{I}|\psi}\Rightarrow \psi(q_{F}) =\int dq_{I}\braket{q_{F}|q_{I}}\psi(q_{I})
\end{equation}
とできるためである。また位置表示の基底ベクトルは正規直交なので(\ref{f-kn})の初期条件として
\begin{equation}
  K(q_{F},t_{F};q_{I},t_{I})|_{t_{F}=t_{I}}=\delta (q_{F}-q_{I})
\end{equation}
が得られる。さらに(\ref{f-kn})に状態の完全性を挿入すれば
\begin{equation}
  K(q_{F},t_{F};q_{I},t_{I})=\int dq' \braket{q_{F},t_{F}|q',t'} \braket{q',t'|q_{I},t_{I}}=\int dq' K(q_{F},t_{F};q',t')(q',t';q_{I},t_{I})
\end{equation}
を満たすことがわかる。あるいは正規直交化されたエネルギー固有関数系$\nkakko{\varphi_{n}(q)}$を用いれば
\begin{align}
  K(q_{F},t_{F};q_{I},t_{I})&=\braket{q_{F}|e^{-\frac{i}{\hbar}(t_{F}-t_{I}H)}|q_{I}}\\
  &=\sum_{n}\bra{q_{F}}e^{-\frac{i}{\hbar}(t_{F}-t_{I})H}\ket{\varphi_{n}} \braket{\varphi_{n}|q_{I}}\\
&=\sum_{n}e^{-\frac{i}{\hbar}(t_{F}-t_{I})E_{n}}\braket{q_{F}|\varphi_{n}} \braket{\varphi_{n}|q_{I}}\\
&=\sum_{n}e^{-\frac{i}{\hbar}(t_{F}-t_{I})E_{n}}\varphi_{n}(q_{F})\varphi_{n}^{*}(q_{I})
\end{align}
とFeynman核を表すことができる。
\subsection{経路積分表示}
\begin{thebibliography}{99}
  \bibitem{QFT-kashiwa} 柏太郎, 新板 演習場の量子論, サイエンス社[2009]   
  \bibitem{QFT-sakamoto} 坂本眞人, 場の量子論(II), 裳華房[2021(第2版)]   
\end{thebibliography}
\end{document}
