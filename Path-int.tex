\documentclass[10pt]{jsarticle}
\usepackage[dvipdfmx]{graphicx}
\usepackage{braket}
\usepackage{bm,latexsym,amsmath,amssymb,amsfonts,mathrsfs}
\usepackage{tensor}
\usepackage{comment}
%\usepackage{physics}
\usepackage[dvipdfmx,linktocpage=true]{hyperref}
\usepackage{pxjahyper}
\usepackage{color}
\usepackage{here}
\usepackage{tikz}
\usetikzlibrary{arrows.meta}
\usetikzlibrary{bending}
\input{colordvi.tex}

\setcounter{tocdepth}{3}%目次にsubsubsectionまで入れる


\newcommand{\kakko}[1]{\left(#1 \right)} %丸括弧()
\newcommand{\kkakko}[1]{\left[ #1 \right]} %角括弧[]
\newcommand{\nkakko}[1]{\left\{ #1 \right\}} %波括弧{}
\newcommand{\p}[1]{\left| #1 \right|} %縦棒||
\newcommand{\henbi}[2]{{\frac{\partial#1}{\partial#2}}} %偏微分
\newcommand{\bi}[2]{{\frac{d #1}{d #2}}} %微分

\newcommand{\del}{{\partial}} %偏微分の∂
\newcommand{\skakko}[1]{$\ll$#1$\gg$}
\newcommand{\vc}[1]{\overrightarrow{#1}}% 矢印ベクトル
\newcommand{\vct}[1]{\bm{#1}}%太字ベクトル
\newcommand{\sint}[1]{\int\mathcal{D}#1\,}%経路積分
\newcommand{\pms}[1]{\mathcal{D}#1\,}%経路積分の測度
\newcommand{\Tr}{\mathrm{Tr\,}}
\newcommand{\tr}{\mathrm{tr\,}}

%\newcommand{\Res}{\mathrm{Res}}
\renewcommand{\Im}{\mathrm{Im\,}}
\renewcommand{\Re}{\mathrm{Re\,}}


\DeclareMathOperator*{\Res}{Res}
\DeclareMathOperator*{\ext}{ext}

\renewcommand{\theenumi}{\Alph{enumi}}
\renewcommand{\labelenumi}{(\theenumi)}

\numberwithin{equation}{section}%数式番号を(n.m)の形で書く。

\topmargin=0.0in
\headsep=0.0in
\headheight=0.0in
\oddsidemargin=-0.22in
\evensidemargin=-0.22in
\textwidth=6.5in
\textheight=9.0in
\title{経路積分の方法}
\author{あけみ@homuotaku}
\date{\today}
\pagestyle{empty}
\begin{document}
\maketitle
経路積分についてまとめる。主に\cite{QFT-kashiwa}と\cite{QFT-sakamoto}に拠っている。
参考資料や私の気分によって、このノートでもノーテーションのブレがあるかもしれない。
\tableofcontents
\section{Feynmanの考え}
2重スリット実験の帰結では、電子銃から放たれ、2つのスリットを通った粒子のスクリーン上での状態は、確率密度の重ね合わせで実現する。
そのスリットを3重にしたら、あるいはスリットをもう1枚増やしたらどうなるかというと、スクリーンに到達した電子の状態は、これもまたそれらの重ね合わせで表される。
ではスリットの穴と枚数をどんどん増やしていき、ついに全面が穴となり、また電子銃とスクリーンの間に隙間なくそのスリットが敷き詰められた場合を考える。
結局のところ、電子銃とスクリーンの間に何も無いのと同じなのだが、前述の議論を参考にすれば「あらゆる可能な経路を通る重ね合わせ」でスクリーン上の状態が実現するといえる。
言い換えれば、始状態と終状態の間の遷移確率振幅は、その2つを結ぶすべての経路の重ね合わせで与えられる、ということになる。
\section{量子力学において}
はじめにハミルトニアン$H$は時間並進の生成子であり、$U(t)=\exp{(-\frac{i}{\hbar}tH)}$は$t$だけ時間をシフトするユニタリ演算子であることを確認する。

時間並進の演算子$U(t,t_{0})$として、状態$\ket{\psi(t_{0})}$から$\ket{\psi(t)}$に時間発展させることは以下のように書ける。
\begin{equation}
\label{hatten}  \ket{\psi(t)}=U(t,t_0)\ket{\psi(t_{0})}
\end{equation}
この演算子は
\begin{align}
 \label{32} i\hbar \henbi{}{t}U(t,t_{0})&=H(t)U(t,t_{0}),\quad U(t_{0},t_{0})=I\\
  \label{33} U(t_{2},t_{1})U(t_{1},t_{0})&=U(t_{2},t_{0}),\quad(t_{2}>t_{1}>t_{0})
\end{align}
を満たす。これを示すために、Schrödinger方程式
\begin{equation}
  i\hbar \henbi{}{t}\ket{\psi(t)}=H(t)\ket{\psi(t)}
\end{equation}
を考える。(\ref{hatten})を代入すれば
\begin{equation}
  i\hbar \henbi{}{t}U(t,t_0)\ket{\psi(t_{0})}=H(t)U(t,t_0)\ket{\psi(t_{0})}
\end{equation}
ここで$\ket{\psi(t_0)}$は任意なので、例えば位置表示で$\ket{\psi(t_{0})}=\ket{q}$をし、さらに両辺$\bra{q}$をかけて積分することで完全性によって(\ref{32})が成立する。次にハミルトニアンが時間に依存しないとき
\begin{equation}
  U(t,t_{0})=\exp\kakko{-\frac{i}{\hbar} (t-t_0)H}
\end{equation}
と書けることを示す。時間を$N$等分することを考え、(\ref{32})の時間微分について無限小の$\Delta t=(t-t_{0})/N$を用いて書き直す。
\begin{equation}
  i\hbar \frac{U(t+\Delta t,t)-U(t,t)}{\Delta t}=H(t)U(t,t)
\end{equation}
$U(t,t)=I$なので
\begin{equation}
   U(t+\Delta t,t)=I-\frac{i}{\hbar}\Delta t H(t)
\end{equation}
さらに$t_{j}=t_{j-1}+\Delta t(j=1,2\cdots,N)$と書けば
\begin{equation}
  U(t_{j},t_{j-1})=I-\frac{i}{\hbar}\Delta t H(t_{j})
\end{equation}
これと(\ref{33})を繰り返し用いれば、
\begin{equation}
  U(t,t_{0})=\lim_{N\to \infty} U(t,t_{N-1})\cdots U(t_{1},t_{0})=\lim_{N\to \infty}\kakko{ I-\frac{i}{\hbar}\Delta t H(t) }\cdots \kakko{I-\frac{i}{\hbar}\Delta t H(t_{1})}
\end{equation}
と書ける。なお、$t_{N}=t$である。いまハミルトニアンは時間依存しない状況を考えているので$H(t)=H$と書いて、$\Delta t=(t-t_{0})/N$と表示すれば
\begin{equation}
  U(t,t_{0})=\lim_{N\to \infty}\kakko{I-\frac{i}{\hbar}\frac{t-t_0}{N}H }^{N}=\exp\kakko{-\frac{i}{\hbar}(t-t_{0})H}
\end{equation}
と書ける。
\subsection{Feynman核}
ここでは{\bf Feynman核}について議論する。Feynman核は
\begin{equation}
 \label{f-kn} K(q_{F},t_{F};q_{I},t_{I})=\braket{q_{F},t_{F}|q_{I},t_{I}}=\braket{q_{F}|e^{-\frac{i}{\hbar}(t_{F}-t_{I})H}|q_{I}}
\end{equation}
で与えられ、経路積分において基本となるものである。
これは時刻$t_{I}$、位置$q_{I}$の粒子が時刻$t_{F}$で位置$q_{F}$に見出される確率振幅に対応する。
Feynman核を用いて波動関数表示について
\begin{equation}
  \psi(q_{F},t_{F})=\int dq_{I}K(q_{F},t_{F};q_{I},t_{I})\psi(q_{I},t_{I})
\end{equation}
と書ける。これは状態を位置表示で
\begin{equation}
  \ket{\psi}=\int dq_{I}\ket{q_{I}}\braket{q_{I}|\psi}
\end{equation}
と書けることから両辺$\bra{q_{F}}$をかけて
\begin{equation}
  \braket{q_{F}|\psi}=\int dq_{I}\braket{q_{F}|q_{I}}\braket{q_{I}|\psi}\Rightarrow \psi(q_{F}) =\int dq_{I}\braket{q_{F}|q_{I}}\psi(q_{I})
\end{equation}
とできるためである。また位置表示の基底ベクトルは正規直交なので(\ref{f-kn})の初期条件として
\begin{equation}
  K(q_{F},t_{F};q_{I},t_{I})|_{t_{F}=t_{I}}=\delta (q_{F}-q_{I})
\end{equation}
が得られる。さらに(\ref{f-kn})に状態の完全性を挿入すれば
\begin{equation}
  K(q_{F},t_{F};q_{I},t_{I})=\int dq' \braket{q_{F},t_{F}|q',t'} \braket{q',t'|q_{I},t_{I}}=\int dq' K(q_{F},t_{F};q',t')K(q',t';q_{I},t_{I})
\end{equation}
を満たすことがわかる。あるいは正規直交化されたエネルギー固有関数系$\nkakko{\varphi_{n}(q)}$を用いれば
\begin{align}
  K(q_{F},t_{F};q_{I},t_{I})&=\braket{q_{F}|e^{-\frac{i}{\hbar}(t_{F}-t_{I}H)}|q_{I}}\notag \\
  &=\sum_{n}\bra{q_{F}}e^{-\frac{i}{\hbar}(t_{F}-t_{I})H}\ket{\varphi_{n}} \braket{\varphi_{n}|q_{I}}\notag \\
&=\sum_{n}e^{-\frac{i}{\hbar}(t_{F}-t_{I})E_{n}}\braket{q_{F}|\varphi_{n}} \braket{\varphi_{n}|q_{I}}\notag \\
&=\sum_{n}e^{-\frac{i}{\hbar}(t_{F}-t_{I})E_{n}}\varphi_{n}(q_{F})\varphi_{n}^{*}(q_{I})
\end{align}
とFeynman核を表すことができる。
\subsection{経路積分表示}
さて、このFeynman核を用いて、1で述べた「あるゆる可能な経路を通る重ね合わせ」という考えが実現されていることを見る。
今考えている時間間隔を$N$等分して、各時刻を
\begin{equation}
  t_{n}=t_{I}+n\Delta t, \quad \Delta t=\frac{t_{F}-t_{I}}{N}
\end{equation}
と表示する。($n=0,1,\cdots N$)当然$t_{0}=t_{I},t_{N}=t_{F}$である。次に、Feynman核に、各時刻$t_{1},t_{2},\cdots ,t_{N-1}$における完全系
\begin{equation}
  \int dq_{n}\ket{q_{n},t_{n}}\bra{q_{n},t_{n}}=I
\end{equation}
を順次挿入して、以下のように変形する。
\begin{align}
  K(q_{F},t_{F};q_{I},t_{I})=& \braket{q_{F},t_{F}|q_{I},t_{I}} \notag \\
    =& \bra{q_{F},t_{F}}\kakko{\int dq_{N-1}\ket{q_{N-1},t_{N-1}}\bra{q_{N-1},t_{N-1}}}\notag  \\
  & \times \kakko{\int dq_{N-2}\ket{q_{N-2},t_{N-2}}\bra{q_{N-2},t_{N-2}}}\cdots\notag \\\
  & \times kakko{\int dq_{1}\ket{q_{1},t_{1}}\bra{q_{1},t_{1}}} \ket{q_{I},t_{I}}\notag \\
  =& \int dq_{1}dq_{N-1}\braket{q_{F},t_{F}|q_{N-1},t_{N-1}}\braket{q_{N-1},t_{N-1}|q_{N-2},t_{N-2}}\notag \\
 \label{fkd}&\times \braket{q_{n},t_{n}|q_{n-1},t_{n-1}} \cdots \braket{q_{2},t_{2}|q_{1},t_{1}}\braket{q_{1},t_{1}|q_{I},t_{I}}
\end{align}
この式において、$\braket{q_{n},t_{n}|q_{n-1},t_{n-1}}$は時刻$t_{n-1}$で位置$q_{n-1}$にいた粒子が時刻$t_{n}=t_{n-1}+\Delta t$で位置$q_{n}$に見出される遷移確率振幅である。
よって、一番右から見て行けば、$(q_{I},t_{I})\to(q_{1},t_{1})\to \cdots \to (q_{N-1},t_{N-1})\to (q_{F},t_{F})$と辿ったときの振幅に対応する。
さらに、始状態と終状態は固定されており、中間状態$n=1,2,\cdots N-1$の部分はすべて積分されているので、これがすべての可能な経路を足し合わせることを表している。
これではじめに述べた考えが実現されていることとなる。
\subsubsection{相空間における表示}
ここでは相空間における表示を導く。
すなわち、Feynman核を
\begin{equation}
  K(q_{F},t_{F};q_{I},t_{I})=\int_{q_I}^{q_{F}} \pms{q}\pms{p}\exp\kkakko{\frac{i}{\hbar}\int_{t_{I}}^{t_{F}}dt\kakko{p(t)\dot{q}(t)-H(p,q)} }
\end{equation}
とする表示を求める。
まず、
\begin{equation}
  \braket{q_{n},t_{n}|q_{n-1},t_{n-1}}=\braket{q_{n}|e^{-\frac{i}{\hbar}\Delta H}|q_{n-1}}
\end{equation}
を評価する。ハミルトニアン内部には演算子である$q,p$が含まれることに注意せよ。ハミルトニアンは
\begin{equation}
  H(p,q)=\frac{p^{2}}{2m}+V(q)
\end{equation}
の形で与えられるとする。いま、$\Delta t$は十分小さいとすると、時間並進について
\begin{equation}
\label{bisho}  e^{-\frac{i}{\hbar}\Delta H}=I-\frac{i}{\hbar}\Delta t H(p,q)+O(\Delta t^{2})
\end{equation}
と書ける。これを用いれば
\begin{align}
  \braket{q_{n}|e^{-\frac{i}{\hbar}\Delta H}|q_{n-1}}&=\braket{q_{n}|\kakko{I-\frac{i}{\hbar}\Delta t H(p,q)+O(\Delta t^{2})}|q_{n-1}}\notag \\
  &=\braket{q_{n}|\kakko{I-\frac{i}{\hbar}\Delta t H(p,q)}|q_{n-1}}+O(\Delta t^{2})\notag \\
  &=\braket{q_{n}|\kakko{I-\frac{i}{\hbar}\Delta t H(p,q)}\kakko{\int dp_{n}\ket{p_{n}}\bra{p_{n}}}|q_{n-1}}+O(\Delta t^{2})\notag \\
  &=\int dp_{n} \kakko{I-\frac{i}{\hbar}\Delta t H(p_{n},q_{n})}\braket{q_{n}|p_{n}}\braket{p_{n}|q_{n-1}}+O(\Delta t^{2})
\end{align}
と変形できる。ここで、下から2番目の等号で完全系を挿入し、最後の等号において$\bra{q_{n}},\ket{p_{n}}$の固有値$q_{n},p_{n}$で書かれている。
すなわち、演算子の部分がすべてc数で書き換えられたことになる。
\begin{equation}
  \braket{q|p}=\frac{1}{\sqrt{2\pi \hbar }}e^{\frac{i}{\hbar}pq}
\end{equation}
と書けることを用いて、再び(\ref{bisho})で指数関数で表せば
\begin{equation}
  \braket{q_{n},t_{n}|q_{n-1},t_{n-1}}=\int \frac{dp_{n}}{2\pi \hbar} \exp\kkakko{\frac{i}{\hbar}\Delta t \kakko{ p_{n}\frac{q_{n}-q_{n-1}}{\Delta t} -H(p_{n},q_{n}) }}+O(\Delta t^{2})
\end{equation}
を得る。これを(\ref{fkd})に用いる。
\begin{equation}
 \label{sokukan1} K(q_{F},t_{F};q_{I},t_{I})=\kakko{\prod_{n=1}^{N-1}\int dq_{n}}\kakko{\prod_{n=1}^{N}\int \frac{dp_{n}}{2\pi \hbar}}\exp\kkakko{\frac{i}{\hbar}\sum_{n=1}^{N}\Delta t \kakko{ p_{n}\frac{q_{n}-q_{n-1}}{\Delta t} -H(p_{n},q_{n}) }}
\end{equation}
$N\to \infty$と取れば$t_{n}$は連続的な変化をするとみなせるので、$q_{n},p_{n}\to q(t),p(t)$と書けて、また
\begin{equation}
 \label{ryakki1} \sum_{n=1}^{N}\Delta t \to \int_{t_{I}}^{t_{F}} dt , \quad \frac{q_{n}-q_{n-1}}{\Delta t}\to \dot{q}(t)
\end{equation}
と表せる。さらに
\begin{equation}
\label{ryakki2}\kakko{\prod_{n=1}^{N-1}\int dq_{n}}\kakko{\prod_{n=1}^{N}\int \frac{dp_{n}}{2\pi \hbar}}=\int_{q_{I}}^{q_{F}} \pms{q}\pms{p}
\end{equation}
と表記すれば、{\bf 相空間における経路積分表示}を得られる。
\begin{equation}
   \label{sokukan} K(q_{F},t_{F};q_{I},t_{I})=\int_{q_I}^{q_{F}} \pms{q}\pms{p}\exp\kkakko{\frac{i}{\hbar}\int_{t_{I}}^{t_{F}}dt\kakko{p(t)\dot{q}(t)-H(p,q)} }
\end{equation}
\subsubsection{配位空間における表示}
ここでは配位空間における経路積分表示
\begin{equation}
  K(q_{F},t_{F};q_{I},t_{I})=\int_{q_I}^{q_{F}} \pms{q}\exp\kakko{\frac{i}{\hbar}S[q]}
\end{equation}
を導く。なお、ここで$S[q]$は
\begin{equation}
  S[q]=\int_{t_{I}}^{t_{F}} dt \, L(q(t),\dot{q}(t))=\int_{t_{I}}^{t_{F}} dt\kakko{\frac{1}{2}m\dot{q}(t)-V(q)}
\end{equation}
と表される作用積分である。
前節で得た(\ref{sokukan})を見ると、指数の肩にラグランジアンの形式\footnote{ハミルトニアンはLegendre変換で$H=p\dot{q}-L$と書けた。}
が現れている。ただし、ここで導入している$p$と$q$は独立であるため、ラグランジアンそのものではない。
しかし、具体的に運動量積分することでラグランジアンの表示にすることが可能だ。
(\ref{sokukan1})で$p_{n}$に関する積分を実行する。
ハミルトニアンの形から、$p_{n}$の2次式となっていることに注目すると、平方完成してGauss積分の形へと持ち込める。ここだけ中学数学!?(暁美ほむら理論)
\begin{equation}
  p_{n}\frac{q_{n}-q_{n-1}}{\Delta t} - \frac{p_{n}^{2}}{2m}-V(q_{n})=-\frac{1}{2m}p_{n}'^{2}+\frac{m}{2}\kakko{\frac{q_{n}-q_{n-1}}{\Delta t}}^{2}-V(q_{n})
\end{equation}
ここで
\begin{equation}
  p_{n}'=p_{n}-m\frac{q_{n}-q_{n-1}}{\Delta t}
\end{equation}
と置いた。この変換した$p_{n}'$で積分を実行することで
\begin{equation}
  K(q_{F},t_{F};q_{I},t_{I})=\kakko{\frac{m}{2\pi\hbar i \Delta t}}^{N/2} \kakko{\prod_{n=1}^{N-1}\int dq_{n}}\exp\kkakko{\frac{i}{\hbar}\sum_{n=1}^{N}\Delta t \kakko{\frac{m}{2}\kakko{\frac{q_{n}-q_{n-1}}{\Delta t}}^{2}-V(q_{n}) }}
\end{equation}
を得て、前節同様に$N\to \infty$として(\ref{ryakki1})と(\ref{ryakki2})で書き換えれば
\begin{equation}
  K(q_{F},t_{F};q_{I},t_{I})=\int \pms{q}\exp\kkakko{\frac{i}{\hbar}\int_{t_{I}}^{t_{F}}dt\kakko{\frac{m}{2}\dot{q}^{2}-V(q) }}
\end{equation}
と書かれる。ここで全体の比例係数は積分測度$\pms{q}$に入れて再定義した。
\begin{thebibliography}{99}
  \bibitem{QFT-kashiwa} 柏太郎, 新板 演習場の量子論, サイエンス社[2009]   
  \bibitem{QFT-sakamoto} 坂本眞人, 場の量子論(II), 裳華房[2021(第2版)]   
\end{thebibliography}
\end{document}
